% compile using the following sequence of commands:
% pdflatex writeup
% bibtex writeup
% pdflatex writeup
% pdflatex writeup

% basic setup:
\documentclass[12pt, titlepage]{article}
\usepackage[utf8]{inputenc}
\usepackage[letterpaper, margin=1in]{geometry}
\usepackage{setspace}
\usepackage{amsmath}

% set up bibliography:
\usepackage[backend=bibtex, style=phys]{biblatex}
\addbibresource{citations.bib}

% change title sizes:
\usepackage{titlesec}
\titleformat*{\section}{\large\bfseries}
\titleformat*{\subsection}{\bfseries}

% change table of contents formatting:
\usepackage{tocloft}
\renewcommand{\cfttoctitlefont}{\large\bfseries}
\renewcommand{\cftsecfont}{\normalsize}
\renewcommand{\cftsubsecfont}{\normalsize}

% title page information:
\title{\Large Nucleation and Growth During the Crystalization \\
		of Amorphous Te Thin Films \\
		\bigskip
	\normalsize MSE 130: Experimental Materials Science and Design}
\author{\normalsize Jonathan Lee \\
	\normalsize Department of Materials Science and Engineering \\
	\normalsize University of California, Berkeley}
\date{\normalsize September 28th, 2020}

\begin{document}

\maketitle

\doublespacing

\setcounter{page}{2}

\tableofcontents

\newpage

\section{Abstract}

\section{Introduction}

novel electronic devices such as flexible displays will require the fabrication of semiconductors on noncrystalline substrates...the low melting points of such substrates (such as polymers etc) means that such processes must function on a low thermal budget...the semiconductor tellurium (Te) is an attractive candidate due to its relatively low melting point (449.5 degrees C)...this allows it to be deposited amorphously at low temperature, after which it may be crystallized near room temperature...additionally, the narrow band gap of bulk Te (0.32 eV) allows for straightforward bandgap engineering through electron confinement...depositing and crystallizing Te in a thin film allows its bandgap to be tuned to optimum values for a given application, such as FETs

the amorphous-deposition-crystallization manufacturing process may be understood in the context of classical Becker-Doering nucleation theory, which models the physical system as a distribution of atomic clusters.  The free energy change ($\Delta G$) of such a cluster is:
%
	\begin{equation}
		\Delta G(i) \approx -g_{c \rightarrow a}i + \gamma i^{2/3}
	\label{eqn:deltaG}
	\end{equation}
%
where $i$ is the number of atoms in the cluster, -$g_{c \rightarrow a}$ is the atomic free energy of the amorphous-to-crystalline transition, and $\gamma$ is a factor convolving the cluster surface energy and the surface-area-to-volume ratio.  Becker-Doering theory then uses steady-state and detailled-balance assumptions in order to derive the area nucleation rate $\dot{N}$:
%
	\begin{equation}
		\dot{N} \approx \dot{N_0} 
		\exp\left[-\frac{\Delta E_{attach} + \Delta G(i_c)}{k_T T}\right]
	\label{eqn:N_dot}
	\end{equation}
%
where $i_c$ is the critical cluster size and $\Delta E_{attach}$ is the activation energy for an atom to jump across the amorphous-crystalline interface.  The assumption of a spherical nucleus (Equation \ref{eqn:deltaG}) motivates a critical nucleation energy of $\Delta G(i_c) = \frac{4 \gamma^3}{27g_{c \rightarrow a}^2}$.  The units of $\dot{N}$ are $m^{-2}s^{-1}$.

The crystallization process may be additionally described through Johnson-Mehl-Avrami-Kolmogorov theory in the 2D limit, wherein a nucleus grows to span the thickness of the film faster than the next nuclus can appear.  In such a limit, the area fraction transformed ($\alpha$) is described by the JMAK equation:
%
	\begin{equation}
		\alpha = 1 - \exp
		\left[ -\frac{\pi}{3}\dot{N}v^2t^3 \right]
	\label{eqn:alpha}
	\end{equation}
%
where the growth velocity (in units of $\frac{m}{s}$) follows an Arrhenius dependence:
%
	\begin{equation}
		v \approx v_0 \exp \left[ - \frac{\Delta E_{attach}}{k_B T} \right]
	\label{eqn:v}
	\end{equation}
%
Finally, JMAK theory produces an expression for the average number of grains nucleated per unit area:
%
	\begin{equation}
		\rho = \left(\frac{3}{\pi}\right)^{\frac{1}{3}}
		\Gamma \left[ \frac{4}{3} \right]
		\left( \frac{\dot{N}}{v} \right)^{\frac{2}{3}}
	\label{eqn:rho}
	\end{equation}
%
where $\Gamma \left[ \frac{4}{3} \right] \approx 0.8929795...$ is the Euler Gamma function.

Taken collectively, B-D theory and JMAK theory yield a straightforward method for calculating $\dot{N}$, $v$, and their associated activation energies and prefactors.  An amorphous Te thin film may be monitored during its anneal in order to measure the crystalline area fraction as a function of time, which may be fitted to Equation \ref{eqn:alpha} to calculate $\dot{N}v^2$.  Afterwards, the grain density may be assessed and Equation \ref{eqn:rho} may be employed to calculate $\dot{N}/v$.  These values may be trivially solved for $\dot{N}$ and ${v}$.

Repeating the procedure at different annealing temperatures allows the Arrhenius forms (Equations \ref{eqn:v} and \ref{eqn:N_dot}) to be fit, thus providing estimates of the activation energies governing nucleation and growth in the low-temperature Te manufacturing process.

The equations presented in this section represent the core results of B-D and JMAK theory.  For a detailled discussion of their derivations and assumptions, as well as a demonstration of the validity of the 2D limit, see Reference \cite{chrzan:2020}.


\section{Experimental Procedure}

An Edwards 306 Thermal Evaporator was loaded with Te source pellets of 99.999\% purity from Sigma-Aldrich.  Te was deposited on a substrate at a deposition rate of approximately 10 \r{A}/sec.  Film thicknesses, as controlled by a quartz crystal microbalance, ranged from 4 to 20 nm.  Since the substrate was maintained at -80\textdegree{C}, the deposited film was amorphous; this was verified by x-ray diffraction.

Six deposited films were selected for experimental analysis.  Immediately after deposition, each film was heated to its target annealing temperature (10\textdegree{C}, 15\textdegree{C}, 20\textdegree{C}, 25\textdegree{C}, 30\textdegree{C}, or 35\textdegree{C}).  Optical microscopy was used to observe the crystallization at an imaging rate of 2.87 frames/sec, with the intent that these data be used to assess the area fraction transformed (Equation \ref{eqn:alpha}).

After annealing, the samples were allowed to return to room temperature, then examined via transmission electron microscope.  Bright field and dark field images of a 140 \textmu{m} x 105 \textmu{m} region were obtained, with the intent that they be used to determine the films' grain density (Equation \ref{eqn:rho}).  Due to their finer microstructure, additional images of a zoomed-in 140 \textmu{m} x 105 \textmu{m} region were collected for the 20\textdegree{C}, 25\textdegree{C}, 30\textdegree{C}, and 35\textdegree{C} samples.


\section{Results}

obtaining and fitting fraction transformed curves

assessing grain density from microstructural images

calculating parameters and estimating free energies/critical nuclei

there is a book  
living inside your chest  
with dilated instructions on how to make a safe landing.


\section{Discussion}

discussion of validity of free energy and critical nuclei result

discussion of validity of grain counting algorithm

discussion of improvements to the estimation procedure and attempt to derive gamma

\subsection{Comments Upon Validity of Area Transformed Data}

As described in Subsection 4.1, the area fraction transformed ($\alpha$) was calculated for each optical frame simply as the average intensity of that frame.  The entire $\alpha$ curve for the sample was subsequently rescaled.  After this analysis was complete, however, it was observed that several of the videos comtain a faint lighting gradient, which may have negatively impacted the accuracy of the calculation.  Specifically, if the sample were improperly illuminated such that the crystalline regions received more or less light on average, then this would produce a systematic error in the results $\alpha$-curves.  Such a scenario becomes progressively more likely as the annealing temperature is lowered and the crystalline phase nucleates in a less-distributed manner.

A different analysis pipeline that addresses the issue would be to binarize and adaptively threshold each frame, thereby counting the fraction of pixels deemed ``bright enough'' to be crystalline based on local lighting conditions.  This, however, has the disadvantage of being sensitive to a dirtied optical lens; care would need to be taken to ignore pixels occluded by dirt or dust on the microscope. 

\subsection{Comments Upon Accuracy of Grain-Counting Algorithm}

Perhaps the most unsound decision made in this analysis was to avoid quantifying inaccuracies in the grain-counting algorithm of Subsection 4.2.  As shown in Figure \ref{fig:miscount}, the algorithm produces miscounts if it fails to find a local maximum for a given grain.  A more rigorous approach, such as that taken in Reference \cite{campbell:2018}, would involve aggressively overestimating the number of grains, after which a supplementary algorithm might be employed to merge adjacent watershed cells deemed similar enough to comprise a single grain.

The possibility of testing the algorithm against computer-generated training data was considered.  In this scenario, bright field and dark field images would be digitally generated for a variety of grain densities, then fed to the algorithm in order to quantify its standard error.  This error would then be translated to error in $\dot{N}/v$ and propagated through the remainder of the calculations.  The main obstacle to this analysis \textit{in silico} is that there is no way of guaranteeing that such artificial electron micrographs would possess qualities similar to real ones; the procedure would quantify the algorithm's error with respect to the sythetic data set only.

In this respect, the results of the grain-counting algorithm should only by viewed as a loose estimate, albeit with greator rigor and consistency than manual counting.

	\begin{figure}[h]
		\centering
		\includegraphics[width=1.0\linewidth]{miscount_10C.png}
		\caption{}
		\label{fig:miscount}
	\end{figure}

\subsection{Validity of Obtained Crystallization Parameters}

As evidenced by the graph in Figure \ref{fig:cluster_size}, the results of the parameter calculations of Subsection 4.3 do not truly yield an estimate for $g_{c \rightarrow a}$ and $i_c$, but rather a methodology for predicting them based on an estimate of $\gamma$.  From its lower limit of 2.683 atoms, the critical cluster size may be made arbitrarily large by changing the input $\gamma$, and a literature review yielded no methodology (beyond large-scale atomic simulations) for calculating the energy of the crystalline-amorphous interface.  These parameters should also be regarded with skepticism, given the assumption of a spherical nucleus with isotropic properties; in reality, Te manifests a trigonal crystal structure with a highly-anisotropic Wulff shape.  Refining the calculation to reflect these properties would likely require reengineering the Becker-Doering and JMAK results from the ground up.

Despite these reservations regarding the fundamental parameters of the crystalline-amorphous transformation, the Arrhenius laws for $\dot{N}$ and $v$ appear to fit the data moderately well, and as such may be used to predict the samples' annealing behavior.  The large standard errors of the prefactors in Table \ref{parameters_result} are a byproduct of fitting the data in $1/T$-ln coordinates; the actual coefficients of determination for the linear fit are $R^2_{\dot{N}} = 0.970$ and $R^2_v = 0.894$.  Most of the uncertainty appears to stem from the grain counts at 25 \textdegree{C} and 30 \textdegree{C}, which appear to be above and below the trend set by the other data, respectively.



\section{Conclusions}

\section{Acknowledgments}

\printbibliography[heading=bibnumbered]

\end{document}

%test\cite{chrzan:2020}
%test\cite{campbell:2018}
%test\cite{chan:2020}
%test\cite{jain:2018}
