An Edwards 306 Thermal Evaporator was loaded with Te source pellets of 99.999\% purity from Sigma-Aldrich.  Te was deposited on a substrate at a deposition rate of approximately 10 \r{A}/sec.  Film thicknesses, as controlled by a quartz crystal microbalance, ranged from 4 to 20 nm.  Since the substrate was maintained at -80\textdegree{C}, the deposited film was amorphous, as later verified by x-ray diffraction.\cite{chrzan:2020}

Six deposited films were selected for experimental analysis.  Immediately after deposition, each film was heated to its target annealing temperature (10\textdegree{C}, 15\textdegree{C}, 20\textdegree{C}, 25\textdegree{C}, 30\textdegree{C}, or 35\textdegree{C}).  Optical microscopy was used to observe the crystallization at an imaging rate of 2.87 frames/sec,\cite{chrzan:2020} with the intent that these data be used to assess the area fraction transformed (Equation \ref{eqn:alpha}).

After annealing, the samples were allowed to return to room temperature, then examined via transmission electron microscope.  Bright field and dark field images of a 140 \textmu{m} x 105 \textmu{m} region were obtained, with the intent that they be used to determine the films' grain density (Equation \ref{eqn:rho}).  Additional images of a zoomed-in 140 \textmu{m} x 105 \textmu{m} region were collected for the 20\textdegree{C}, 25\textdegree{C}, 30\textdegree{C}, and 35\textdegree{C} samples.\cite{chrzan:2020}  These higher-temperature films were observed to possess finer microstructures, so the additional images were to serve as a backup in case the grain density proved too difficult to quantify at the original magnification.
